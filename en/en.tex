\documentclass[a4paper, 12pt]{moderncv}
\usepackage[top=3cm, bottom=3cm, left = 2.5cm, right = 2.5cm]{geometry}
\usepackage{setspace}
\usepackage[utf8]{inputenc}

\moderncvstyle{banking}
\moderncvcolor{blue}

\renewcommand{\familydefault}{\sfdefault}

\name{Pedro}{Folloni Pesserl}
\address{MY ADDRESS}{Curitiba/PR}{Brazil}
\phone[mobile]{+55 (41) 998962301}
\email{fpesserl7@gmail.com}
\homepage{linkedin.com/in/pedropesserl}
\homepage{github.com/pedropesserl}

\begin{document}
\makecvtitle

\small{Computer Science undergraduate at Federal University of Paraná (UFPR).
Interested in various areas of the computational knowledge, from Algorithm
Theory to the study of Operational Systems and Computer Networks.}

\section{Experience}
\vspace{1pt}
\subsection{\small{Scholarship -- Emerging Leaders of the Americas Program (ELAP)}}
\textit{Université de Moncton, sep/2024--dec/2024}\hspace{190pt}\textbf{Moncton/CA}

\vspace{3pt}
I was accepted as a research intern at Université de Moncton, where I am currently developing research for the PRIMe (Perception, Robotics and Intelligent Machines) laboratory, about using deep learning diffusion models to segment fire in wildfire pictures.
\vspace{4pt}

\subsection{\small{Scholarship -- Centro de Computação Científica e Software Livre (C3SL)}}
\textit{Federal University of Paraná, apr/2024--currently}\hspace{175pt}\textbf{Curitiba/BR}

\vspace{3pt}
As a scholarship holder in the C3SL laboratory at UFPR, I help to create and mantain projects that aim to improve Brazil's digital sovereignty and benefit the Brazilian society in general. In C3SL, I have developed the following activities:
\vspace{4pt}
\begin{itemize}
    \item{\textbf{Database management}: In the C3SL Database Team, I work with the PostgreSQL and ClickHouse DBMSs, helping to mantain the databases used by other projects in the laboratory.}
    \item{\textbf{System administration}: Use of the many Linux virtual machines in C3SL's cluster to manage services and test software tools. Use of Docker to containerize those services.}
\end{itemize}
\vspace{4pt}

\subsection{\small{Scholarship -- PET Computação UFPR}}
\textit{Federal University of Paraná, jun/2022--feb/2024}\hspace{175pt}\textbf{Curitiba/BR}

\vspace{3pt}
I was a member of the PET Computação group at UFPR, and I took part in several
projects aimed to improve the University and the external society. There, I
practiced my creativity and team-working skills on a daily basis, in activities
such as:
\vspace{4pt}
\begin{itemize}
    \item{\textbf{Web Development}: Use of the web framework Ruby on Rails for
          development of the project ADEGA, a tool for visualizing statistics
          related to the history of the undergraduate programs.}
    \item{\textbf{Treasury}: Management of resources earned for investment in
          equipment for the group.}
    \item{\textbf{Event planning}: I planned a trip with more than 30 students
          to participate in the event Roadsec 2023, in São Paulo.}
    \item{\textbf{Student groups}: I gave classes in student groups for
          disciplines of the Computer Science program.}
\end{itemize}

\vspace{4pt}
\subsection{\small{Research}}
\begin{itemize}
    \item{\cventry{sep/2024--dec/2024}{PRIMe (Perception, Robotics and Intelligent Machines) - UMoncton}
        {Forest Fire Image Segmentation}{Moncton/CA}{}
        {I am researching deep learning diffusion models for automatic segmentation of
        fire in images. Advisor: Prof. Moulay Akhloufi, PhD, Department of
        Informatics -- UMoncton.}}
    \item{\cventry{sep/2022--apr/2023}{Federal University of Paraná}
        {Scientific Initiation}{Curitiba/BR}{}
        {I participated in a mathematics research on generalizing arithmetic functions
        to categories. Topics: category theory and number theory. Advisor: Dr. Eduardo
        Hoefel, Department of Mathematics -- UFPR.}}
\end{itemize}

\vspace{4pt}
\subsection{\small{Volunteering}}
\begin{itemize}
    \item{\cventry{jul/2023}{Roadsec 2023}{Voluntary Roadie}{São Paulo/BR}{}
            {\vspace{3pt}I worked as an assistant to the main stage, collecting
            audience questions for the speakers, and remained in communication
            with the team to resolve any issues throughout the event.}}
        \vspace{4pt}
\end{itemize}
\vspace{4pt}

\section{Education}
\vspace{1pt}
\subsection{\small{Academic Qualification}}
\vspace{2pt}
\begin{itemize}
    \item{\cventry{2022--2026 (expected)}{Federal University of Paraná}
        {Bachelor in Computer Science}{Curitiba/BR}{}{}}
\end{itemize}
\vspace{4pt}

\section{Abilities}
\vspace{4pt}
\begin{itemize}
    \item{\textbf{Programming languages:} C, Bash, x86-64 Assembly, Pascal, VHDL,
        \TeX, C++, Ruby, JavaScript.}
    % \vspace{4pt}
    \item{\textbf{Other languages and frameworks:} HTML, CSS, SQL, Bootstrap,
        Ruby on Rails, Raylib.}
    \vspace{4pt}
    \item{\textbf{DataBase Management Systems:} PostgreSQL, ClickHouse, SQLite3.}
    \vspace{4pt}
    \item{\textbf{Software tools:} Git, Github/Gitlab, Linux-based Operating
        Systems, Docker, GDB, Google Workspace, LibreOffice/MS Office.}
    \vspace{4pt}
    \item{\textbf{Soft Skills:} good communication, conflict resolution, team management,
        quick learning.}
    \vspace{4pt}
    \item{\textbf{Other:} good writing, abstraction and generalization skills in
        mathematics, interest in scientific research.}
\end{itemize}
\vspace{4pt}

\section{Languages}
\vspace{4pt}
\begin{itemize}
    \item{\textbf{Portuguese} -- Fluent}
    \item{\textbf{English} -- Fluent}
    \item{\textbf{Spanish} -- Basic}
    \item{\textbf{German} -- Basic}
\end{itemize}
\vspace{4pt}

\section{Interests and extracurricular activities}
\vspace{4pt}
\begin{itemize}
    \item{I participated in the first phase of the SBC Programming Marathon, in
        september of 2023, which took place in the Centro Politécnico campus of
    the Federal University of Paraná.}
    \vspace{4pt}
    \item{I took a German class in the Center of Languages and Interculturality of
        UFPR (CELIN).}
    \vspace{4pt}
    \item{I created a Tetris-based game in the language C to play in the Linux
        terminal, available in \url{https://github.com/pedropesserl/tetric}.}
    \vspace{4pt}
    \item{I developed a physics simulator of gravity applied to bodies in
        space, also in C, with a GUI written with the library Raylib. Available
    in \url{https://github.com/pedropesserl/nbody}.}
\end{itemize}


\end{document}
