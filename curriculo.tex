\documentclass[a4paper, 12pt]{moderncv}
\usepackage[top=3cm, bottom=3cm, left = 2.5cm, right = 2.5cm]{geometry}
\usepackage[brazilian]{babel}
\usepackage{setspace}
\usepackage[utf8]{inputenc}

\moderncvstyle{banking}
\moderncvcolor{blue}

\renewcommand{\familydefault}{\sfdefault}

\name{Pedro}{Folloni Pesserl}
\address{MEU ENDEREÇO}{Curitiba/PR}{Brasil}
\phone[mobile]{+55 (41) 998962301}
\email{fpesserl7@gmail.com}
\homepage{linkedin.com/in/pedropesserl}
\homepage{github.com/pedropesserl}

\begin{document}
\makecvtitle

\small{Estudante de Bacharelado em Ciência da Computação pela Universidade
Federal do Paraná. Interessado nas mais diversas áreas do conhecimento
computacional, desde a Teoria de Algoritmos até o estudo de Sistemas
Operacionais e das Redes de Computadores.}

\section{Experiência}
\vspace{4pt}
\begin{itemize}
    \item{\cventry{jul/2023}{Roadsec 2023}{Roadie Voluntário}{São Paulo/SP}{}
        {\vspace{3pt}Atuei como auxiliar do palco principal, na coleta de dúvidas
        da plateia para os palestrantes, e me mantive em comunicação com a equipe
        para resolver eventuais contratempos pela duração do evento.}}
    \vspace{4pt}
    \item{\cventry{2022--Atualmente}{Universidade Federal do Paraná}
        {Bolsista -- PET Computação UFPR}{Curitiba/PR}{}
        {\vspace{3pt}Sou integrante do grupo PET Computação e participante de
        diversos projetos voltados ao progresso da Universidade e da sociedade
        externa. Lá, exercito diariamente minha criatividade e capacidade de trabalho
        em equipe.}}
\end{itemize}
\vspace{4pt}

\section{Educação}
\vspace{1pt}
\subsection{\small{Qualificação Acadêmica}}
\vspace{2pt}
\begin{itemize}
    \item{\cventry{2022--2026 (previsto)}{Universidade Federal do Paraná}
        {Bacharelado em Ciência da Computação}{Curitiba/PR}{}{}}
\end{itemize}
\vspace{4pt}

\section{Habilidades}
\vspace{4pt}
\begin{itemize}
    \item{\textbf{Linguagens de programação:} C, Bash, Assembly x86, Pascal, VHDL,
        \TeX, C++, Ruby e JavaScript.}
    \vspace{4pt}
    \item{\textbf{Demais linguagens e frameworks:} HTML, CSS, Ruby on Rails.}
    \vspace{4pt}
    \item{\textbf{Ferramentas de software:} Git, Github/Gitlab, Sistemas
        Operacionais Linux, GDB, Google (Drive, busca avançada, Planilhas,
        Documentos, Apresentações, Formulários, Agenda etc.), LibreOffice/MS
        Office.}
    \vspace{4pt}
    \item{\textbf{Soft Skills:} boa comunicação, mediação de conflitos, gerenciamento
        de equipes, aprendizado rápido.}
    \vspace{4pt}
    \item{\textbf{Outros:} boa escrita, capacidade de abstração e generalização na
        matemática, interesse na pesquisa científica.}
\end{itemize}
\vspace{4pt}

\section{Idiomas}
\vspace{4pt}
\begin{itemize}
    \item{\textbf{Português} -- Fluente}
    \item{\textbf{Inglês} -- Fluente}
    \item{\textbf{Espanhol} -- Básico}
    \item{\textbf{Alemão} -- Básico}
\end{itemize}
\vspace{4pt}

\section{Interesses e atividades extracurriculares}
\vspace{4pt}
\begin{itemize}
    \item{Participei da 1\textsuperscript{a} fase da Maratona de Programação
        SBC, em setembro de 2023, sediada no campus Centro Politécnico da Universidade
        Federal do Paraná.}
    \vspace{4pt}
    \item{Faço parte de tutorias de matérias do curso de Ciência da Computação,
        organizadas pelo PET Computação da UFPR para ajudar alunos com eventuais
        dificuldades nessas disciplinas.}
    \vspace{4pt}
    \item{Atualmente estou cursando a disciplina de Alemão I no Centro de Línguas
        e Interculturalidade da UFPR (CELIN).}
\end{itemize}


\end{document}
