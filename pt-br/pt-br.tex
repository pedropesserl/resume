\documentclass[a4paper, 12pt]{moderncv}
\usepackage[top=3cm, bottom=3cm, left = 2.5cm, right = 2.5cm]{geometry}
\usepackage[brazilian]{babel}
\usepackage{setspace}
\usepackage[utf8]{inputenc}

\moderncvstyle{banking}
\moderncvcolor{blue}

\renewcommand{\familydefault}{\sfdefault}

\name{Pedro}{Folloni Pesserl}
\address{MEU ENDEREÇO}{Curitiba/PR}{Brasil}
\phone[mobile]{+55 (41) 998962301}
\email{fpesserl7@gmail.com}
\homepage{linkedin.com/in/pedropesserl}
\homepage{github.com/pedropesserl}

\begin{document}
\makecvtitle

\small{Estudante de Bacharelado em Ciência da Computação pela Universidade
Federal do Paraná. Interessado nas mais diversas áreas do conhecimento
computacional, desde a Teoria de Algoritmos até o estudo de Sistemas
Operacionais e das Redes de Computadores.}

\section{Experiência}
\vspace{1pt}
\subsection{\small{Bolsista -- PET Computação UFPR}}
\textit{Universidade Federal do Paraná, 2022--Atualmente}\hspace{182pt}\textbf{Curitiba/PR}

\vspace{3pt}
Sou integrante do grupo PET Computação e participante de diversos projetos
voltados ao progresso da Universidade e da sociedade externa. Lá, exercito
diariamente minha criatividade e capacidade de trabalho em equipe em
atividades como:
\vspace{4pt}
\begin{itemize}
    \item{\textbf{Desenvolvimento Web}: Uso do framework web Ruby on Rails para
          desenvolvimento do projeto ADEGA, uma ferramenta de visualização de
          estatísticas relacionadas às grades acadêmicas dos cursos de graduação da
          UFPR.}
    \item{\textbf{Administração de Sistemas}: Administração da rede
          conjunta de computadores dos membros do PET.}
    \item{\textbf{Tesouraria}: Gerenciamento dos recursos recebidos para investimento
          em equipamentos para o grupo.}
    \item{\textbf{Organização de eventos}: Organizei uma excursão com mais de
          30 alunos dos cursos de Ciência da Computação e Informática Biomédica
          para participarmos do evento Roadsec 2023 em São Paulo.}
    \item{\textbf{Tutorias}: Faço parte de tutorias de matérias do curso de Ciência
          da Computação, organizadas para ajudar alunos com eventuais dificuldades.}
\end{itemize}

\vspace{4pt}
\subsection{\small{Pesquisa}}
\begin{itemize}
    \item{\cventry{set/2022--abr/2023}{Universidade Federal do Paraná}
        {Iniciação Científica}{Curitiba/PR}{}
        {Participei de uma pesquisa sobre generalização de funções aritméticas em
         categorias. Tópicos: teoria de categorias e teoria de números. Orientador:
         Prof. Dr. Eduardo Hoefel, Departamento de Matemática -- UFPR.}}
\end{itemize}

\vspace{4pt}
\subsection{\small{Voluntariado}}
\begin{itemize}
    \item{\cventry{jul/2023}{Roadsec 2023}{Roadie Voluntário}{São Paulo/SP}{}
            {\vspace{3pt}Atuei como auxiliar do palco principal, na coleta de dúvidas
                da plateia para os palestrantes, e me mantive em comunicação com a equipe
            para resolver eventuais contratempos pela duração do evento.}}
        \vspace{4pt}
\end{itemize}
\vspace{4pt}

\section{Educação}
\vspace{1pt}
\subsection{\small{Qualificação Acadêmica}}
\vspace{2pt}
\begin{itemize}
    \item{\cventry{2022--2026 (previsto)}{Universidade Federal do Paraná}
        {Bacharelado em Ciência da Computação}{Curitiba/PR}{}{}}
\end{itemize}
\vspace{4pt}

\section{Habilidades}
\vspace{4pt}
\begin{itemize}
    \item{\textbf{Linguagens de programação:} C, C++, Bash, Assembly x86,
        Pascal, VHDL, \TeX, Ruby e JavaScript.}
    \vspace{4pt}
    \item{\textbf{Demais linguagens e frameworks:} HTML, CSS, Bootstrap, Ruby on Rails.}
    \vspace{4pt}
    \item{\textbf{Ferramentas de software:} Git, Github/Gitlab, Sistemas
        Operacionais Linux, GDB, Google (Drive, busca avançada, Planilhas,
        Documentos, Apresentações, Formulários, Agenda etc.), LibreOffice/MS
        Office.}
    \vspace{4pt}
    \item{\textbf{Soft Skills:} boa comunicação, mediação de conflitos, gerenciamento
        de equipes, aprendizado rápido.}
    \vspace{4pt}
    \item{\textbf{Outros:} boa escrita, capacidade de abstração e generalização na
        matemática, interesse na pesquisa científica.}
\end{itemize}
\vspace{4pt}

\section{Idiomas}
\vspace{4pt}
\begin{itemize}
    \item{\textbf{Português} -- Fluente}
    \item{\textbf{Inglês} -- Fluente}
    \item{\textbf{Espanhol} -- Básico}
    \item{\textbf{Alemão} -- Básico}
\end{itemize}
\vspace{4pt}

\section{Interesses e atividades extracurriculares}
\vspace{4pt}
\begin{itemize}
    \item{Participei da 1\textsuperscript{a} fase da Maratona de Programação
        SBC, em setembro de 2023, sediada no campus Centro Politécnico da Universidade
        Federal do Paraná.}
    \vspace{4pt}
    \item{Atualmente estou cursando a disciplina de Alemão I no Centro de Línguas
        e Interculturalidade da UFPR (CELIN).}
    \vspace{4pt}
    \item{Criei um jogo baseado em Tetris na linguagem C feito para jogar no terminal
        Linux, disponível em \url{https://github.com/pedropesserl/tetric}.}
\end{itemize}


\end{document}
